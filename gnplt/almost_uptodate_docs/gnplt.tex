\documentstyle[titlepage,psfig,11pt]{article}
\topmargin 0in
\textheight 8.5in
\oddsidemargin 0in
\textwidth 6.1in

\input psfig.sty

\begin{document}

\centerline{\large \bf NOTES ON USING GNPLT (Ver 1)}
\vskip 0.4cm
\centerline{ John Conway, Onsala Space Observatory}.
\noindent
\vskip 2cm

\noindent {\bf 1) Introduction} 
\vskip 0.5cm

  The Tcl/Tk program GNPLT is provided with the FS release 
  tool to help examine the performance 
  of your telescope, check the accuracy of the 
  calibration parameters stored in the control .rxg file 
  and to  update  this information if needed. The general 
  philosophy of the FS calibration is described in a separate 
  document.

  GNPLT takes two sets of sources inputs. First is information 
  collected in a log file containing  runs of  ONOFF usually 
  taken as part of an AQUIRE run.
  Specifically from this log the information on the ONOFF VAL line 
  including the  cal source name and  measurements of SEFD(Jy), Tcal(Jy) etc. 
  are used. The second source of inputs  are the 
  calibration parameters   from the relevant .rxg file 
  for the particular observing band  (that is the Tcal(K) versus freq, dpfu,
   gain curve polynomial). When GNPLT is started 
   these cal parameters are  copied from the relevant .rxg file into an  
   internal 'working file'. As calibration parameters are fitted the 
   values of Tcal vs freq, DPFU and gain curves within the working 
   file can be  updated. When the programme is finally 
   exited the user can choose to update the .rxg file stored on disk 
   with  the cal parameters currently in the working file. 

   Within GNPLT many different quantities  can be plotted.
   It is useful to be aware of three separate types of quantities.  
   Quantities like SEFD(JY), Tcal(Jy) are the MEASUREMENT QUANTITIES   
   taken direct from the input log file, these never change. 
   Other quantities like  Tcal(K) are CAL QUANTITIES,  these always 
   take their value from what is stored  in the current working file. 
   Finally  are the DERIVED QUANTITIES like gain(K/Jy) which are derived from
   a combination of a MEASURED quantity (in this case Tcal(Jy) and a 
   CAL QUANTITY (e,g Tcal(K)). 

   The following sections indicate give step by step instructions on 
  how calibration parameters 
   can be checked and updated if needed. 

% First the pointing performance
%   of the telescope should be investigated (see Section 2). Next GNPLT 
%   is started, data read in and edited (section 3). Within GNPLT we 
%   should next   plot antenna gain versus elevation to check the 
%   overall amplitude scale and shape of the gain curve: both 
%   of which can be updated if needed. Next (see Section 4) we 
%   should plot the Tcal(K)  versus frequency and update the Tcal(K) 
%   table if needed. Finally when  we have good fits between data and 
%   model in both gain versus elevation and Tcal(K) versus Frequency we can exit GNPLT and update  the .rxg file on disk. This .rxg will be used to 
%   form calibration data for each VLBI experiment to be  sent to the 
%    astronomer. If the steps of running GNPLT below are followed this 
%    calibration should be accurate.

\vskip 1cm

\noindent {\bf 2) Check Pointing} 
\vskip 0.5cm

  Calibration data will only worth plotting in GNPLT if the pointing 
  performance of your telescope is good. If as part of your ACQUIRE  
  FIVEPT data was collected the FS Tcl/Tk program 
  PDPLT can be used to check pointing. If the pointing offsets
  are large but are consistent you can go ahead and analyse the ONOFF
  data, but in this case you will have to update the pointing model for
  your telescope to ensure good calibration in the VLBI runs.

\vfill
\eject

\noindent {\bf 3) Starting GNPLT and loading a Logfile} 
\vskip 0.5cm

   Start the program by typing 'gnplt'. The log file can be 
   including on the command line or inside GNPLT you can 
   click on FILE-NEW, select input file from menu and then click
   to load the log file containing ONOFF data.

\vskip 0.5cm

\noindent {\bf 4) Selecting data and plotting quantities.}

\vskip 0.5cm

   Click EDIT and then select which combination of LCP, RCP 
   and frequency channels to plot.
 
   Choose a x-axis type from ITEMS(X-AXIS), e.g elevation. 
   Then choose a y-axis type from ITEMS(Y-AXIS). 
   In this drop down menu are listed a set of 
   data that can be plotted such as  azimuth, elevation, gain 
   compression, tsys, SEFD, Tcal(Jy) Tcal(K), Tcal(ratio) - all of these 
   are the measurements quantities in the log file {\it except Tcal(K) 
   which is a derived quantity  calculated from the   measured Tcal(K) 
   and the current DPFU and Gaincurve}.  It may be useful when 
   examining your antenna performance  to plot quantities like SEFD vs 
   elevation. 


   The Y-AXIS drop down menu  also contains an  'assumed items' 
   button which plots the currently assumed calibration quantities like 
   DPFU etc. The final button on the Y-AXIS menu is 'Gain'. The 
   Gain is the antenna gain (K/Jy), calculated from the
   Tcal(Jy) measured in the log and other inputs. When selecting 
   this input you are further prompted to choose one of three options
   for determining how exactly the Gain is calculated. These three 
   options are

\vskip 0.5cm
 

\noindent (A) Assume the Tcal(K) for each point given from the 
   input .rxg file. This is the normal mode of operation.
   The plotted  Gain (K/Jy)  = Tcal(K)/Tcal(Jy).
   If we think the Tcal(K) in the .rxg is approximately correct 
   then this option should be selected.

\vskip 0.5cm

\noindent   (B) Tcal(K)  from user input. If you have an improved 
    value Tcal(K) from hot/cold load measurements  you can enter 
    it here. The programme then uses this Tcal(K) together 
    with Tcal(Jy) measurements in the log to compute the gain for each point.
    Use this option if you are sure of the new Tcal(K) measurement 
    and wish to change the other cal parameter to be consistent with this 
    and the  ONOFF measurements.

\vskip 0.5cm

\noindent   (C) Specify gain for elevation range. This calculates the 
    gain using the stored Tcal(K) in the working file but 
    then a scaling factor is found such that the resulting gains 
    match a specified value over an elevation range. The 
    Tcal(K) in the current parameters are all multiplied 
    by this scaling factor. Use this option if you are sure of the 
    your new gain  measurement  and wish to change the other cal 
    parameter to be consistent with this  and the  ONOFF measurements.

\vskip 0.5cm
 
    We are also given the option to apply atmosphere 
    opacity corrections before plotting the gain. This 
    option is not yet implemented. The use of the Y-axis 
    'gain' option is described in more detail 
    in Section ?  below.

\vskip 1cm

\noindent {\bf 5) Editing out bad data.}

\vskip 0.5cm

    Before we can test whether the gain parameters in 
    the .rxg file are valid we must delete 
    bad data. 

    Selecting as the 'Y-axis' type  gain compression is a 
    good way to select out good data.  All measurements should have a 
    gain compression of around 
    1  (gain compression is the change in counts firing the noise
    cal on source divided by the change in counts firing the cal off
    sources). If there is no intermittent RFI, there are no 
    slewing problems and the electronics is linear then all the 
   measurements should be around 1.
 
  To edit out bad data, press on left mouse button to mark the bottom left 
  corner of rectangle    then drag the cursor and release to form top-right 
  corner. The plot axes changes and  all points outside this rectangle are 
  flagged (marked red) and not used in future fits. The axis scaling also 
  ignores flagged points. The status of any point can be changed from flagged
  to un flagged or vice-versa by clicking on the point. You can plot other 
  quantities on the y-axis and do further editing.


\vskip 1cm

\noindent {\bf 6) Checking Gain Curve and overall cal scale.}

\vskip 0.5cm

\noindent {\bf 6.1) Plot First polarisation.}

\vskip 0.5cm

   The next steps include determining whether the model gain curve
   and overall amplitude scale are correct. If not we must update 
   the calibration parameters.  

   Use EDIT to select one polarisation (LCP), it might 
   also be useful to select one frequency channel in the 
   middle of the band. Now  select X-AXIS - elevation, for Y-AXIS 
   select  GAIN. For plotting the gain curve and testing whether it 
   fits the measurements choose the option of 'Tcal(K) 
   from input control file'. Click on TOOLS and then GAIN CURVE to 
   plot  the current gain curve. This model gain is calculated  from 
   the present value of DPFU and the gain polynomial in  the 'working file'. 
   The gain curve is plotted as a dotted line on top of the measurements.
   Check whether the overall shape and height of the  curve are correct.
   If the shape of the gain curve is correct but not the height
   this suggests an overall scale error in  the amplitude  calibration (see
   section 6.1). If the shape is incorrect this implies an inaccurate gain curve
   polynomial (see Section 6.2).

\vskip 0.5cm

\noindent {\bf 6.2) Updating Amplitude cal scale.}
 
\vskip 0.5cm

   If the amplitudes of the model gain and the masurements are 
   inconsistent this means that the Tcal(Jy) 
   predicted from the .rxg file and that measured from sources
   of known flux density using ONOFF are not consistent.
   In order to get accurate calibration for VLBI  you will  
   need to change either the DPFU or scale the  Tcal(K) vs freq table. 
   To update one of these click on TOOLS - FIT-TO and then select  
   'DPFU' or  'scale the Tcal(K)'. 

\vskip 0.5cm

   {\bf Updating DPFU} You should select to update 
   DPFU if you believe your Tcal(K) is correct but that 
   your effective telescope gain stored in the .rxg 
   file may be in error.  You would choose this 
   option if for instance the Tcal(K) has recently been 
   measured absolutely by hot/cold measurements. You might 
   also prefer this option if  the  plotted points are close to 
   what you expect for the antennas gain given 
   the diameter and efficiency  (dpfu  = efficency*geometrical area 
   (m$^2$)/2760). Using this  option a better fit between data and model 
   is obtained  by rescaling the gain curve to better fit the measured 
   points. After re-fitting for DPFU  a new solid curve is plotted over 
   the measured points and the  value of DPFU is updated in 
   the 'working file'. 
   
\vskip 0.5cm

   {\bf Updating Tcal(K)}  You should choose this  option if you
   suspect the effective noise cal has changed (for instance 
   the  noise tube has been re-installed with a new cross 
   coupler or attenuation in the path). You might 
   also consider choosing  this option if you think the gain given by 
   the model is closer to what you expect for your telescope than  
   the measured points.  This option produces a better fit between 
   data and model by rescaling the data to fit the model gain curve. 
   After fitting for 'scale Tcal(K)' the Tcal(K) versus
   frequency table stored in the 'working file' version of the 
   cal parameters are all rescaled. Note that even though the scale 
   correction might  have been found using data from a single frequency 
   channel the same rescaling is applied to the Tcal(K) values 
   at all frequencies. New values of the antenna gain for each measurement 
   are calculated using the updated Tcal versus frequency table
   and are re-plotted. If the model gain curve is plotted it should 
   fit the re-plotted points.


\vskip 0.5cm
\noindent {\bf 6.3) Updating Gain Polynomial and DPFU.}
\vskip 0.5cm

    If the shape of 
    model gaincurve versus elevation does not fit  the data, then 
    under TOOLS-FIT TO you should select option 'GAIN CURVE and DPFU'.
    After fitting  a solid line representing the new 
    model gain is plotted, this should now fit the data. 
    After running this option both the gain curve polynomial and the 
    DPFU for this polarisation are updated in the 'working file'.


\vskip 0.5cm
\noindent {\bf 6.4) Fitting second polarisation.}
\vskip 0.5cm

    After having fitted the data for LCP then go back to the  EDIT 
    menu and plot the data (for one frequency channel) now selecting  RCP.
    Again using TOOLS-GAIN CURVE plot the current model gain versus elevation
    If the amplitude of this model and the data are different select 
    TOOLS-FIT TO. selecting 'fit DPFU' or 'Scale Tcal(K)'. The Field
    System assumes the same gain curve polynomial for both RCP and LCP.
    If as expected the same polynomial applies to both polarisations
    there should be no need to choose the option to update  gain curve. 
    Note that if  you do choose to update the gaincurve, the gaincurve
    polynomial is updated for both RCP and LCP.  

\vskip 1cm

\noindent {\bf 7) Check and update Tcal(K) versus frequency}.

 
     
\end{document}








